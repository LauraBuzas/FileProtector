\chapter{Introduction}
	\paragraph{}
	Privacy is one of the things humanity has been keen on in the last few decades. To achieve safety software-wise, as a first thing, a person would like to protect personal data. 
	
	\paragraph{}
	What is important to keep in mind regarding data security is a matter of three concepts, namely availability, confidentiality and integrity, each one them having a high importance. Data availability means that it is available at any time, in case the user tries to access it.
	
	\paragraph{}
	Confidentiality is all about data that cannot be accessed without adequate authorization. A solution to take into account when it comes to confidentiality would be encryption. This is a proper solution, regarding the fact that a user's data will be available to read only for him.
	
	\paragraph{}
	Integrity means that user's data cannot be modified without authorization. In case integrity is violated, the data would be in an inconsistent state, not what the user is expecting. This would also render the data unavailable because the user does not have access to the old data anymore.
	
	\paragraph{}
	To achieve data safety you can consider having your data encrypted, but even with it, the integrity problem would still be open.   
	
	\paragraph{}
	The main objective of this paper is to describe the process of design and implementation of a solution that tackles the integrity problem mentioned above. The designed solution is composed of: a minifilter driver that contains the monitoring and protection logic, a DLL that is an abstract interface over the minifilter functionality, and a Windows Desktop application built on top of WPF (Windows Presentation Foundation). 
	
	\paragraph{}
	The second chapter contains Linux and Windows built in protection file mechanisms as well as a product similar to the file protector. This chapter also goes into detail about Linux mechanisms and a potential design for the file protector in order to make it run on UNIX systems. Then, it describes the needed Windows theoretical concepts, such as the I/O manager and the Filter Manager, in order to implement the described product. Moreover, it goes into the specifics of the implementation as well as the testing approach. 
	
	\paragraph{}
	Third chapter describes the further work that can be done to add more functionality to the current application. These improvements contain both performance enhancements and extended functionality.  
	
	\paragraph{}
	Last chapter is a conclusion of the paper, summing up the main ideas. Here the need for such a product is restated and its main goals and the means of achieving them are concluded.
		
	\section{Personal Contribution}
	\paragraph{}
	The main contribution consists of designing an SDK (Software Development Kit) that can be integrated by third-parties. My work consists of providing a documented interface for integrators in order to easily manage file protection rules that are based on Windows Kernel-level mechanisms.
	
	\paragraph{}
	The purpose of this is to provide an abstraction over a kernel module that contains all the logic for monitoring and blocking unauthorized file access. Moreover, this interface should be easy to use for third-party vendors.
	
	